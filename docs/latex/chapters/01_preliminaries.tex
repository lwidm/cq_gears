\section{Preliminaries}

\subsection{Notation and Symbols}

\subsubsection*{Gear Geometry}
\begin{tabular}{@{}lll@{}}
    $m$         & Module                    & \textit{Modul} \\
    $z$         & Number of teeth           & \textit{Zähnezahl} \\
    $\alpha_t$  & Pressure angle (transverse) & \textit{Eingriffswinkel} \\
    $x$         & Profile shift coefficient & \textit{Profilverschiebungsfaktor} \\
\end{tabular}

\subsubsection*{Diameters}
\begin{tabular}{@{}lll@{}}
    $d_p$   & Pitch circle diameter ($d_p = m \cdot z$)     & \textit{Teilkreisdurchmesser} \\
    $d_b$   & Base circle diameter ($d_b = d_p \cos\alpha_t$) & \textit{Grundkreisdurchmesser} \\
    $d_a$   & Addendum (tip) circle diameter                & \textit{Kopfkreisdurchmesser} \\
    $d_f$   & Dedendum (root) circle diameter               & \textit{Fusskreisdurchmesser} \\
    $d^*$   & Arbitrary diameter                            & -- \\
\end{tabular}

\subsubsection*{Tooth Dimensions}
\begin{tabular}{@{}ll@{}}
    $s_0$   & Tooth thickness at pitch circle \\
    $s_b$   & Tooth thickness at base circle \\
    $\gamma$& Half tooth angle at base circle \\
\end{tabular}

\subsubsection*{Curve Parameters and Angles}
\begin{tabular}{@{}ll@{}}
    $\phi$      & Curve parameter (rolling angle) \\
    $\theta$    & Angle with respect to x-axis \\
    $\phi_0$    & Starting parameter value \\
\end{tabular}

\subsubsection*{Vectors and Functions}
\begin{tabular}{@{}ll@{}}
    $\vb{r}_{\text{inv}}(\phi)$     & Involute position vector \\
    $\vb{r}_{\text{hypo}}(\phi)$    & Hypotrochoid position vector \\
    $\vb{T}(\phi)$                  & Unit tangent vector \\
    $\vb{N}(\phi)$                  & Unit normal vector \\
    $\vb{R}(\theta)$                & 2D rotation matrix \\
\end{tabular}

\subsection{Sign Convention}
The symbol $\mypm$ indicates a sign that depends on which flank is being described:
\begin{itemize}
    \item $\textcolor{red}{+}$: Right flank (clockwise rotation)
    \item $\textcolor{red}{-}$: Left flank (counterclockwise rotation)
\end{itemize}

\subsection{Rotation Matrix}
The 2D rotation matrix for angle $\theta$ (counterclockwise positive):
\begin{equation}
    \vb{R}(\theta) = \begin{bmatrix}
        \cos\theta & -\sin\theta \\
        \sin\theta & \cos\theta
    \end{bmatrix}
\end{equation}

