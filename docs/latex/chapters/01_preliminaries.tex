\section{Preliminaries}

\subsection{Notation and Symbols}

\subsubsection*{Gear Geometry}
\begin{tabular}{@{}lll@{}}
    $m$         & Module                    & \textit{Modul} \\
    $m_n$       & Normal module                    & \textit{Normalmodul} \\
    $m_t$       & Transverse module                    & \textit{Axialmodul} \\
    $z$         & Number of teeth           & \textit{Zähnezahl} \\
    $\alpha_t$  & Pressure angle (transverse) & \textit{Eingriffswinkel} \\
    $x$         & Profile shift coefficient & \textit{Profilverschiebungsfaktor} \\
    $h_a^*$     & Addendum coefficient & \textit{Kopfhöhenfaktor} \\
    $c^*$       & Tip clearance factor & \textit{Kopfspielfaktor} \\
    $c$         & Tip clearance & \textit{Kopfspiel} \\
    $p$         & Pith  & \textit{Teilung} \\
    $\alpha_n$  & Normal pressure angle  & \textit{Normaleingriffswinkel} \\
    $\alpha_t$  & Transverse pressure angle   & \textit{Stirneingriffswinkel} \\
    $\beta$     & Helix angle & \textit{Schrägungswinkel} \\
    $\beta_b$   & Base helix angle  & \textit{Grundschrägungswinkel} \\
    $\gamma$    & Lead angle at reference cylinder  & \textit{Steigungswinkel auf dem Teilzylinder} \\
\end{tabular}

\subsubsection*{Diameters}
\begin{tabular}{@{}lll@{}}
    $d_p$   & Pitch (reference) circle diameter ($d_p = m \cdot z$)     & \textit{Teilkreisdurchmesser} \\
    $d_b$   & Base circle diameter ($d_b = d_p \cos\alpha_t$) & \textit{Grundkreisdurchmesser} \\
    $d_a$   & Addendum (tip) circle diameter                & \textit{Kopfkreisdurchmesser} \\
    $d_f$   & Dedendum (root) circle diameter               & \textit{Fusskreisdurchmesser} \\
    $d^*$   & Arbitrary diameter                            & -- \\
    $h_a$   & Addendum from reference pitch circle  & \textit{Zahnkopfhöhe} \\
    $h_f$   & Dedendum from reference circle  & \textit{Zahnfusshöhe}
\end{tabular}

\subsubsection*{Tooth Dimensions}
\begin{tabular}{@{}ll@{}}
    $s_0$   & Tooth thickness at pitch circle \\
    $s_b$   & Tooth thickness at base circle \\
    $\gamma$& Half tooth angle at base circle \\
\end{tabular}

\subsubsection*{Curve Parameters and Angles}
\begin{tabular}{@{}ll@{}}
    $\phi$      & Curve parameter (rolling angle) \\
    $\theta$    & Angle with respect to x-axis \\
    $\phi_0$    & Starting parameter value \\
\end{tabular}

\subsubsection*{Vectors and Functions}
\begin{tabular}{@{}ll@{}}
    $\vb{r}_{\text{inv}}(\phi)$     & Involute position vector \\
    $\vb{r}_{\text{undercut}}(\phi)$    & Undercut curve position vector \\
    $\vb{T}(\phi)$                  & Unit tangent vector \\
    $\vb{N}(\phi)$                  & Unit normal vector \\
    $\vb{R}(\theta)$                & 2D rotation matrix \\
\end{tabular}

\subsection{Basic Gear Geometry}

The fundamental relationships between the gear parameters are summarized below. All formulas follow the (DIN) ISO 21771 standard.

\subsubsection*{Circle Diameters}
Given the module $m$, number of teeth $z$, transverse pressure angle $\alpha_t$, profile shift coefficient $x$, addendum coefficient $h_a^*$, and clearance coefficient $c^*$):

\begin{align}
    d_p &= m \cdot z \\
    d_b &= d_p \cos(\alpha_t) \\
    d_a &= d_p + 2\,h_a = m\cdot(z + 2\,x + 2\cdot h_a^*)\\
    d_f &= d_p - 2\,h_f = m\cdot(z + 2\,x - 2\cdot(h_a^*+c^*))
\end{align}

The addendum and dedendum heights depend on the profile shift:
\begin{align}
    h_a &= (h_a^* + x) \cdot m \\
    h_f &= (h_a^* + c^* - x) \cdot m
\end{align}

For standard spur gears according to (DIN) ISO 21771:
\begin{center}
\begin{tabular}{@{}lll@{}}
    $\alpha_t = \SI{20}{\degree}$ & Transverse pressure angle & \textit{Eingriffswinkel} \\
    $h_a^* = 1$ & Addendum coefficient & \textit{Kopfhöhenfaktor} \\
    $c^* = 0.25$ & Clearance coefficient & \textit{Kopfspielfaktor} \\
\end{tabular}
\end{center}

\subsection{Sign Convention}
The symbol $\mypm$ indicates a sign that depends on which flank is being described:
\begin{itemize}
    \item $\textcolor{red}{+}$: Right flank (clockwise rotation)
    \item $\textcolor{red}{-}$: Left flank (counterclockwise rotation)
\end{itemize}

\subsection{Rotation Matrix}
The 2D rotation matrix for angle $\theta$ (counterclockwise positive):
\begin{equation}
    \vb{R}(\theta) = \begin{bmatrix}
        \cos\theta & -\sin\theta \\
        \sin\theta & \cos\theta
    \end{bmatrix}
\end{equation}

