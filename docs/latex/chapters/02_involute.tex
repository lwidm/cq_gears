\section{The Involute Curve}
\subsection{Geometric Construction}
Figure~\ref{fig:rack-geometry} shows a gear interacting with a rack, with the relevant geometry overlaid:
base circle (\emph{Grundkreis}), pitch circle (\emph{Teilkreis}), addendum and dedendum circles (\emph{Kopf- und Fusskreis}), and the line of action (\emph{Eingriffslinie}).
The involute curve originates from the base circle and defines the tooth flank. As the rack moves linearly, the contact point travels along the line of action, maintaining a constant velocity ratio.
\begin{figure}[htbp]
    \centering
    \includegraphics[width=0.8\textwidth]{de-evolventenverzahnung-zahnstange-eingriff.jpg}
    \caption{Rack and pinion \cite{rack}}
    \label{fig:rack-geometry}
\end{figure}
\begin{figure}[htbp]
    \centering
    \includegraphics[width=0.8\textwidth]{de-evolventenverzahnung-kreis-evolvente-eingriffswinkel-flankenform.jpg}
    \caption{Involute geometry and pressure angle \cite{geometry}}
    \label{fig:involute-geometry}
\end{figure}
The video at \cite{involute} shows how the involute originates in a very intuitive way.

For the mathematical description I will think of the involute as the curve traced by the unwinding of a taut string from the base 

\begin{figure}[htbp]
    \centering
    {
        \href{run:../assets/involute_line.mp4}{
            \includegraphics[width=0.47\linewidth]{../assets/involute_line.png}
        }
        \href{run:../assets/involute_string.mp4}{
            \includegraphics[width=0.47\linewidth]{../assets/involute_string.png}
        }
        % \href{https://github.com/lwidmer/cq_gears/raw/main/docs/assets/involute.mp4}{
        %     \includegraphics[width=0.6\linewidth]{../../output/involute.png}
        % }
        \caption{Two equivalent constructions of the involute of a circle. Left: the involute generated by a straight line rolling without slipping along the base circle. Right: the same curve obtained by unwinding a taut string from the circle.}
    }
    \label{fig:involute_construction}
\end{figure}

Figure~\ref{fig:involute_construction} illustrates two equivalent geometric constructions for the involute. While the rolling-line construction (left) is more directly relevant in a mechanical context, since it reflects the kinematics of a rack rolling without slipping on a gear, the unwinding-string construction (right) showes more clearly that the length of the blue arrow is exactly equal to the arc length spanned by the base circle at the angular parameter~$\phi$.

Keeping this in mind and looking at the decomposition in Figure~\ref{fig:involute_construction}, the parametric equation can be separated into its blue and yellow components:
$$
\begin{bmatrix}
x(\phi) \\
y(\phi)
\end{bmatrix}
= \textcolor[rgb]{0.8,0.8,0}{\frac{d_b}{2} \begin{bmatrix}
\cos(\phi) \\
\sin(\phi)
\end{bmatrix}}
+ \textcolor{blue}{\frac{d_b}{2}\phi \begin{bmatrix}
\sin(\phi) \\
-\cos(\phi)
\end{bmatrix}}
$$

\subsection{Parametric Equation}
\subsection{Parameter \texorpdfstring{$\phi(ud^*)$}{φ(d*)} at a Given Diameter}
\subsection{Involute Angle \texorpdfstring{$\theta(d^*)$}{θ(d*)} at a Given Diameter}
\subsection{Summary}
