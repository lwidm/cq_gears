\section{The Hypotrochoid (Undercut Curve)}
The parametric description of the undercut curve is significantly more complicated. Fortunately Gregelry Bencsik has made a great youtube video~\cite{undercut_YTvideo} which shows in a very intuitive way how the geometry is created and shows a relative simple way one can describe it.

\subsection{Geometric construction}
\begin{figure}[ht]
    \centering
    \includegraphics[width=0.9\textwidth]{../assets/undercut_geometry_YT_screenshot.png}
    \caption{Screenshot of Gragelry Bencsik's video~\cite{undercut_YTvideo} explaining the undercut geomertry. Here $r_p$ refers to the pitch circle radius, $r_d$ to the addendum circle radius (for which I use $r_f$, resp. $d_f$ for the diameter) and alpha the transversal pressure angle (for which I use $\alpha_t$)}
    \label{fig:undercut_video_screenshot}
\end{figure}

Figure~\ref{fig:undercut_video_screenshot} above (and the associated video) show very nicely how the geometry of the undercut is created. Simply put, the rack rolls along the pitch circle (carefull: \textbf{unlike the tooth involute which uses the base circle here the rack rolls along the pitch circle}) and with it the (here right) tip of the rack tooth (shown above with the solid arrow) rotate and translates.

Therefore the undercut can be constructed by placing this solid arrow along the involute created by $d_p$ while keeping the angle of this arrow w.r.t the involute constant.

Again we use $\phi$ as a parameter for this curve however this time a negative $\phi$ will result in the right flank as it is related to the associated clockwise rotating involute used for construction.

\textbf{A major confusion} with Figure~\ref{fig:undercut_video_screenshot} above is the coordinate syste. The math in the following sections creates an undercut curve on the right side of the gear, and not below and therefore you need to treat it as follows:
\begin{itemize}
    \item Positive $x$-direction is downwards
    \item Positive $y$-direction is to the right
    \item I use a $\mypm$ to differenciatiate between the right ($\textcolor{red}{+}$) and left ($\textcolor{red}{-}$) flank.
    \item Negative $\phi$ values will result in a right flank and positve values will result in a left flank.\footnote{Actually both flanks have positive and negative values for $\phi$ as close to the dedendum circle the sing switches. What is ment here is that $\max(\phi)$ is positive or negative.}
\end{itemize}

With this in mind the arrow at $\phi=0$ (which is the state of Figure~\ref{fig:undercut_video_screenshot} above) is:
\begin{equation}
    \vb v = \begin{bmatrix} v_x \\ v_y \end{bmatrix} = \begin{bmatrix} r_d - r_p \\ \mypm r_f \tan(\alpha_t) \end{bmatrix}
\end{equation}
or using $d_f = 2\cdot r_d$ and $d_p = 2 \cdot r_p$
\begin{equation}
    \vb v = \begin{bmatrix} v_x \\ v_y \end{bmatrix} = \begin{bmatrix} \frac{d_f - d_p}{2} \\ \mypm \frac{d_f}{2} \tan(\alpha_t) \end{bmatrix}
\end{equation}


\subsection{Differential Geometry of the Involute}
    \subsubsection{Derivative}
    \subsubsection{Tangent and Normal Vectors}
    \subsubsection{Angle of Tangent/Normal}
\subsection{Shift Vector}
\subsection{Parametric Equation}
\subsection{Parameter \texorpdfstring{$\phi(d^*)$}{φ(d*)} at a Given Diameter}
    \subsubsection{General Case}
    \subsubsection{Special Case \texorpdfstring{$\phi_0 = \phi(d_f)$}{φ₀ = φ(dₓ)}}
\subsection{Summary}
