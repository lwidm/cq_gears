\section{The Undercut Curve}
\label{sec:undercut}
The parametric description of the undercut curve is significantly more complicated. Fortunately, Gergely Bencsik has made a YouTube video~\cite{undercut_profile_shift_YTvideo} which explains the geometry intuitively and presents a relatively simple way to describe it.

\subsection{Geometric construction}
\begin{figure}[ht]
    \centering
    \includegraphics[width=0.9\textwidth]{../assets/undercut_geometry_YT_screenshot.png}
    \caption{Screenshot of Gergely Bencsik's video~\cite{undercut_profile_shift_YTvideo} explaining the undercut geometry. Here $r_p$ refers to the pitch circle radius, $r_d$ to the addendum circle radius (denoted $r_f$, resp. $d_f$ for the diameter, in this document) and alpha the transversal pressure angle (denoted $\alpha_t$)}
    \label{fig:undercut_video_screenshot}
\end{figure}

Figure~\ref{fig:undercut_video_screenshot} (and the associated video) show how the undercut geometry is created. The rack rolls along the pitch circle (careful: \textbf{unlike the tooth involute, here the rack rolls along the pitch circle, not the base circle}) and the tip of the rack tooth (shown with the solid arrow, hereafter referred to as the \textbf{shift vector}) rotates and translates with it.

Therefore the undercut can be constructed by placing this shift vector along the involute created by $d_p$ (hereafter referred to as the \textbf{construction involute}) while keeping the angle of this shift vector w.r.t.\ the involute constant (see Figure~\ref{fig:undercut_generated_video}).

As with the involute, $\phi$ is used as the curve parameter. However, this time a negative $\phi$ results in the right flank, since the construction involute rotates clockwise in this case (opposite to the tooth involute, as shown in Figure~\ref{fig:undercut_video_screenshot}). For the left flank, the opposite holds: positive $\phi$ with a counterclockwise rotating construction involute.


\subsection{Shift Vector}

\textbf{A potential source of confusion} is the coordinate system in Figure~\ref{fig:undercut_video_screenshot}. The math in the following sections creates an undercut curve on the right side of the gear, not below, and therefore the coordinate system should be interpreted as follows:
\begin{itemize}
    \item Positive $x$-direction is downwards
    \item Positive $y$-direction is to the right
    \item A $\mypm$ is used to differentiate between the right ($\textcolor{red}{+}$) and left ($\textcolor{red}{-}$) flank.
    \item Negative $\phi$ values will result in a right flank and positive values will result in a left flank.\footnote{Actually both flanks have positive and negative values for $\phi$ since the sign switches near the dedendum circle. What is meant here is that $\phi_{end}$ is negative (right flank) or positive (left flank), where the curve is drawn starting from the dedendum circle outwards.}
\end{itemize}

With this in mind, the shift vector at $\phi=0$ (the state shown in Figure~\ref{fig:undercut_video_screenshot}) is:
\begin{equation}
    \vb v = \begin{bmatrix} v_x \\ v_y \end{bmatrix} = \begin{bmatrix} r_d - r_p \\ \mypm r_f \tan(\alpha_t) \end{bmatrix}
\end{equation}
or using $d_f = 2\cdot r_d$ and $d_p = 2 \cdot r_p$
\begin{equation}
    \vb v = \begin{bmatrix} v_x \\ v_y \end{bmatrix} = \begin{bmatrix} \frac{d_f - d_p}{2} \\ \mypm \frac{d_f}{2} \tan(\alpha_t) \end{bmatrix}
\end{equation}


\subsection{Differential Geometry of the Involute}
\label{subsec:diff_geom_invo}
Since the shift vector must be moved along the construction involute while keeping the angle w.r.t.\ the construction involute constant (see Figure~\ref{fig:undercut_generated_video}), it needs to be continuously rotated by an angle defined by the construction involute.

The easiest way to do this is to compute the angle of the tangent vector to the construction involute w.r.t.\ the $x$-axis (keeping in mind the rotated orientation of Figure~\ref{fig:undercut_video_screenshot}), since at $\phi=0$ this angle is $0$. This requires computing the derivative of the construction involute w.r.t.\ $\phi$.

\subsubsection{Derivative}
Starting from the parametric involute equation, but using $d_p$ instead of $d_b$ since the construction involute is based on the pitch circle:
\begin{equation}
\vb{r_{inv}}(\phi) =
\begin{bmatrix}
x(\phi) \\
y(\phi)
\end{bmatrix}
= \tfrac{d_p}{2} \begin{bmatrix}
\cos(\phi) \\
\sin(\phi)
\end{bmatrix}
+ \tfrac{d_p}{2}\phi \begin{bmatrix}
\sin(\phi) \\
-\cos(\phi)
\end{bmatrix}
\end{equation}
First the $x$-component is differentiated
\begin{subequations}
\begin{align}
x(\phi) &= \frac{d_p}{2}(\cos(\phi) + \phi \sin(\phi))\\
x'(\phi) &= \frac{d_p}{2}(-\sin(\phi) + \sin(\phi) + \phi\cos(\phi)) \\
&= \frac{d_p}{2}\phi\cos(\phi)
\end{align}
\end{subequations}

Next the $y$-component is differentiated
\begin{subequations}
\begin{align}
y(\phi) &= \frac{d_p}{2}(\sin(\phi) - \phi \cos(\phi))\\
y'(\phi) &= \frac{d_p}{2}(\cos(\phi) - \cos(\phi) + \phi\sin(\phi)) \\
&= \frac{d_p}{2}\phi\sin(\phi)
\end{align}
\end{subequations}

Resulting in:
\begin{equation}
\vb{r_{inv}}'(\phi) = \frac{d_p}{2}\phi\begin{bmatrix}\cos(\phi)\\\sin(\phi)\end{bmatrix}
\end{equation}

\subsubsection{Tangent and Normal Vectors}
To get normalised vectors, the magnitude of the derivative is computed
\begin{subequations}
\begin{align}
\norm{\vb{r_{inv}'}(\phi)} &= \frac{d_p}{2}\abs{\phi}\sqrt{\cos^2(\phi) + \sin^2(\phi)} \\
&= \frac{d_p}{2}\abs{\phi}
\end{align}
\end{subequations}

\paragraph{The unit tangent vector} is computed as:
\begin{subequations}
\begin{align}
\vb{T}(\phi) &= \frac{\vb{r_{inv}}'(\phi)}{\norm{\vb{r_{inv}}'(\phi)}}\\
&= \frac{\phi}{\abs{\phi}}\begin{bmatrix}\cos(\phi)\\\sin(\phi)\end{bmatrix}\\
&= \operatorname{sgn}(\phi) \begin{bmatrix}\cos(\phi)\\\sin(\phi)\end{bmatrix}
\end{align}
\end{subequations}

\paragraph{The unit normal vector} is obtained by rotating the unit tangent by $+90^\circ$.
\begin{equation}
\vb{N}(\phi) = \operatorname{sgn}(\phi)\begin{bmatrix}-\sin(\phi)\\\cos(\phi)\end{bmatrix}
\end{equation}

\subsubsection{Angle of Tangent/Normal}
The angle $\theta_t$ of the tangent relative to the $x$-axis is computed as:
\begin{equation}
\tan(\theta_t) = \frac{y'(\phi)}{x'(\phi)} = \frac{\phi\sin(\phi)}{\phi\cos(\phi)}=\tan{\phi}
\end{equation}

Hence:
\begin{equation}
\theta_t(\phi) = \phi
\end{equation}

Similarly for the angle of the normal vector:
\begin{equation}
\theta_n(\phi) = \phi + \tfrac{\pi}{2}
\end{equation}

\subsection{Parametric Equation}

\begin{figure}[ht]
    \centering
        \iflocal
            \href{run:../assets/hypotroichoid.mp4}{%
                \includegraphics[width=0.47\linewidth]{../assets/hypotroichoid.png}
            }%
        \else
            \href{https://lwidm.github.io/cq_gears/hypotroichoid.mp4}{%
                \includegraphics[width=0.47\linewidth]{../assets/hypotroichoid.png}
            }%
        \fi
    \caption{The shift vector at a later point in the undercut curve construction ($\phi<0$). The image links to a video on GitHub showing the movement of the shift vector along the construction involute.}
    \label{fig:undercut_generated_video}
\end{figure}
As described in previous sections, the shift vector is first rotated by the angle $\phi$ (see Section~\ref{subsec:diff_geom_invo}) and then translated to the correct position of the construction involute:

\begin{subequations}
\begin{align}
    \vb{r_{undercut}}(\phi) &=\vb{R}(\phi)\vb v + \vb{r_{inv}}(\phi)
\\
    &= \underbrace{
            \begin{bmatrix}
                \cos(\phi) & -\sin(\phi) \\
                \sin(\phi) & \cos(\phi)
            \end{bmatrix}}_{\vb R(\phi)}
        \underbrace{
            \begin{bmatrix}
                \frac{d_f - d_p}{2} \\
                \pm\frac{d_f}{2}\tan(\alpha_t)
            \end{bmatrix}}_{\vb v}
        + \underbrace{
            \frac{d_p}{2}
                \begin{bmatrix}
                    \cos(\phi) \\
                    \sin(\phi)
                \end{bmatrix}
            + \frac{d_p}{2}\phi
                \begin{bmatrix}
                    \sin(\phi) \\
                    -\cos(\phi)
                \end{bmatrix}
            }_{\vb{r_{inv}(\phi)}}
\\
    &= \frac{1}{2}\left(
            \begin{bmatrix}
                \cos(\phi) & -\sin(\phi) \\
                \sin(\phi) & \cos(\phi)
            \end{bmatrix}
            \begin{bmatrix}
                d_f - d_p \\
                \pm d_f\tan(\alpha_t)
            \end{bmatrix}
        + d_p
            \begin{bmatrix}
                \cos(\phi) \\
                \sin(\phi)
            \end{bmatrix}
        + d_p\phi
            \begin{bmatrix}
                \sin(\phi) \\
                -\cos(\phi)
            \end{bmatrix}
        \right)
\\[1ex]
    &= \frac{1}{2}\left(
            \begin{bmatrix}
                (d_f - d_p)\cos(\phi) \mp d_f\tan(\alpha_t)\sin(\phi)\\
                (d_f - d_p)\sin(\phi) \pm d_f\tan(\alpha_t)\cos(\phi)
            \end{bmatrix}
            + \begin{bmatrix}
                d_p(\cos(\phi) + \phi\sin(\phi))\\
                d_p(\sin(\phi) - \phi\cos(\phi))\\
            \end{bmatrix}
        \right)
\\[1ex]
    &= \frac{1}{2}
        \begin{bmatrix}
            (d_f - d_p + d_p)\cos(\phi) + d_f\tan(\alpha_t)\sin(\phi) + d_p\phi\sin(\phi) \\
            (d_f - d_p + d_p)\sin(\phi) - d_f\tan(\alpha_t)\cos(\phi) - d_p\phi\cos(\phi) \\
        \end{bmatrix}
\\[1ex]
    &= \frac{1}{2}
        \begin{bmatrix}
            d_f\cos(\phi) \mp d_f\tan(\alpha_t)\sin(\phi) + d_p\phi\sin(\phi) \\
            d_f\sin(\phi) \pm d_f\tan(\alpha_t)\cos(\phi) - d_p\phi\cos(\phi) \\
        \end{bmatrix}
\\[1ex]
    &= \frac{1}{2}\left(
            \begin{bmatrix}
                d_f\\
                \pm d_f\tan(\alpha_t)\\
            \end{bmatrix}
        \cos(\phi)
        +
            \begin{bmatrix}
                \mp d_f\tan(\alpha_t) \\
                d_f \\
            \end{bmatrix}
        \sin(\phi)
        + d_p\,\phi
            \begin{bmatrix}
                \sin(\phi) \\
                - \cos(\phi) \\
            \end{bmatrix}
        \right)
\end{align}
\end{subequations}

\paragraph{Final result} represented \dots

\dots as a single matrix:
\begin{equation}
\vb{r_{undercut}}(\phi) = \frac{1}{2}
        \begin{bmatrix}
            d_f\cos(\phi) \mp d_f\tan(\alpha_t)\sin(\phi) + d_p\phi\sin(\phi) \\
            d_f\sin(\phi) \pm d_f\tan(\alpha_t)\cos(\phi) - d_p\phi\cos(\phi) \\
        \end{bmatrix}
\label{eq:undercut_single_matrix}
\end{equation}

\dots as separated terms:
\begin{equation}
\vb{r_{undercut}}(\phi) = \frac{1}{2}\left(
            \begin{bmatrix}
                d_f\\
                \pm d_f\tan(\alpha_t)\\
            \end{bmatrix}
        \cos(\phi)
        +
            \begin{bmatrix}
                \mp d_f\tan(\alpha_t) \\
                d_f \\
            \end{bmatrix}
        \sin(\phi)
        + d_p\,\phi
            \begin{bmatrix}
                \sin(\phi) \\
                - \cos(\phi) \\
            \end{bmatrix}
        \right)
\label{eq:undercut_separated_terms}
\end{equation}

\dots in compact notation:
\begin{equation}
\begin{gathered}
\vb{r_{undercut}}(\phi) = \frac{1}{2} \left(
    \begin{bmatrix}
        a \\ b
    \end{bmatrix} \cos(\phi)
    + \begin{bmatrix}
        -b \\ a
    \end{bmatrix} \sin(\phi)
    + d_p\,\phi\; \begin{bmatrix}
        \sin(\phi) \\ - \cos(\phi)
    \end{bmatrix}
    \right)
\\
a = d_f \quad\quad b = \pm d_f\tan(\alpha_t)
\label{eq:undercut_compact_notation}
\end{gathered}
\end{equation}


\subsection{Parameter \texorpdfstring{$\phi(d^*)$}{φ(d*)} at a Given Diameter}
\subsubsection{General Case}
Starting from Eq.~\eqref{eq:undercut_compact_notation}, the $x$ and $y$ components are separated:
\begin{subequations}
\begin{align}
x_{undercut}(\phi) &= \frac{1}{2}\left(a\cos\phi - b \sin\phi + d_p\phi\sin\phi\right) \\
y_{undercut}(\phi) &= \frac{1}{2}\left(b\cos\phi + a \sin\phi - d_p\phi\cos\phi\right)
\end{align}
\end{subequations}

With these, the diameter can be determined:
\begin{subequations}
\begin{align}
d^* &\stackrel{!}{=} \sqrt{x_{inv}(\phi)^2 + y_{inv}(\phi)^2} \;\;\cdot 2 \\
&= \sqrt{\frac{1}{4}\cdot \left(\left(a\cos\phi - b \sin\phi + d_p\phi\sin\phi\right)^2 + \left(b\cos\phi + a \sin\phi - d_p\phi\cos\phi\right)^2\right)} \;\;\cdot 2 \\
&= \sqrt{\left(a\cos\phi - b \sin\phi + d_p\phi\sin\phi\right)^2 + \left(b\cos\phi + a \sin\phi - d_p\phi\cos\phi\right)^2}
\label{eq:main_d_star_undercut}
\end{align}
\end{subequations}

Simplifying the terms under the square root in Eq.~\eqref{eq:main_d_star_undercut}:
\begin{subequations}
\begin{align}
\left(a\cos\phi - b \sin\phi + d_p\phi\sin\phi\right)^2 &= a^2\cos^2\phi + b^2\sin^2\phi + d_p^2\phi^2\sin^2\phi \nonumber\\
&\phantom{=} - 2ab\cos\phi\sin\phi + 2ad_p\phi\cos\phi\sin\phi - 2bd_p\phi\sin^2\phi
\\
\left(b\cos\phi + a \sin\phi - d_p\phi\cos\phi\right)^2 &= b^2\cos^2\phi + a^2\sin^2\phi + d_p^2\phi^2\cos^2\phi \nonumber\\
&\phantom{=} + 2ab\cos\phi\sin\phi - 2bd_p\phi\cos^2\phi - 2ad_p\phi\sin\phi\cos\phi
\end{align}
\end{subequations}

Inserting into Eq.~\eqref{eq:main_d_star_undercut}:
\begin{subequations}
\begin{align}
d^* &= \Big[ 
\nonumber\\
    &\phantom{= [ \;\;} \big( a^2\cos^2\phi + b^2\sin^2\phi + d_p^2\phi^2\sin^2\phi 
\nonumber\\
    &\phantom{= [ \;\; \big ( a^2}\textcolor{orange}{\cancel{- 2ab\cos\phi\sin\phi}} \textcolor{blue}{\cancel{+ 2ad_p\phi\cos\phi\sin\phi}} - 2bd_p\phi\sin^2\phi \big) + 
\\
    &\phantom{= [ \;\; } \big( b^2\cos^2\phi + a^2\sin^2\phi + d_p^2\phi^2\cos^2\phi
\nonumber \\
    &\phantom{= [ \;\; \big( b^2} \textcolor{orange}{\cancel{+ 2ab\cos\phi\sin\phi}} - 2bd_p\phi\cos^2\phi \textcolor{blue}{\cancel{- 2ad_p\phi\sin\phi\cos\phi}} \big)
\nonumber \\
    &\phantom{= [ }\Big]^{\tfrac{1}{2}}
\nonumber \\
    &= \sqrt{(a^2 + b^2)\cos^2\phi + (a^2 + b^2) \sin^2\phi +d_p^2\phi^2(\sin^2\phi + \cos^2\phi) - 2bd_p\phi(\sin^2+\cos^2)}
\\
    &= \sqrt{(a^2 + b^2)(\cos^2\phi + \sin^2\phi) +d_p^2\phi^2(\sin^2\phi + \cos^2\phi) - 2bd_p\phi(\sin^2+\cos^2)}
\\
    &= \sqrt{a^2 + b^2 +d_p^2\phi^2 - 2bd_p\phi}
\end{align}
\end{subequations}

Now solving for $\phi(d^*)$
\begin{subequations}
\begin{align}
d^* &= \sqrt{a^2 + b^2 +d_p^2\phi^2 - 2bd_p\phi}\\
{d^*}^2 &= a^2 + b^2 +d_p^2\phi^2 - 2bd_p\phi\\
(d_p^2)\cdot\phi^2 - (2bd_p)\cdot\phi + (a^2 + b^2 - {d^*}^2)& = 0
\end{align}
\end{subequations}

Using the quadratic formula:
\begin{equation*}
ax^2+bx + c = 0 \qq{$\Longrightarrow$} x = \frac{-b \pm \sqrt{b^2 -4ac}}{2a}
\end{equation*}

Results in:
\begin{subequations}
\begin{align}
\phi(d^*) &= \frac{2bd_p \pm \sqrt{4b^2d_p^2 - 4d_p^2(a^2 + b^2 - {d^*}^2)}}{2d_p^2} \\
&= \frac{2bd_p \pm \sqrt{\cancel{4b^2d_p^2} - 4a^2d_p^2 \cancel{- 4b^2d_p^2} + 4{d^*}^2d_p^2}}{2d_p^2} \\
&= \frac{2bd_p \pm 2d_p\sqrt{- a^2 + {d^*}^2}}{2d_p^2}\\
&= \frac{b \pm \sqrt{{d^*}^2- a^2}}{d_p} \\
&= \frac{b}{d_p} \pm \sqrt{\left(\frac{d^*}{d_p}\right)^2- \left(\frac{a}{d_p}\right)^2}
\end{align}
\end{subequations}

Inserting $a=d_f$ and $b=\mypm d_f \tan(\alpha_t)$:
\begin{subequations}
\begin{align}
\phi(d^*) &= \mypm\frac{b}{d_p} \pm \sqrt{\left(\frac{d^*}{d_p}\right)^2- \left(\frac{a}{d_p}\right)^2} \\
&= \mypm\frac{d_f}{d_p}\tan{\alpha_t} \pm \sqrt{\left(\frac{d^*}{d_p}\right)^2- \left(\frac{d_f}{d_p}\right)^2} \\
\end{align}
\end{subequations}

\subsubsection{Separated Right and Left Flank Cases}

\paragraph{Right Flank:}
\begin{equation*}
b = \textcolor{red}{+}d_f\tan(\alpha_t)
\end{equation*}
When dealing with the right flank, the construction involute rotates clockwise, therefore the smaller (usually negative) solution for $\phi$ is relevant. Hence, the square root term is subtracted.
\begin{equation}
    \phi_{right}(d^*) = \frac{d_f}{d_p}\tan{\alpha_t} - \sqrt{\left(\frac{d^*}{d_p}\right)^2 - \left(\frac{d_f}{d_p}\right)^2}
\end{equation}

\paragraph{Left Flank:}
\begin{equation*}
b = \textcolor{red}{-}d_f\tan(\alpha_t)
\end{equation*}
When dealing with the left flank, the construction involute rotates counterclockwise, therefore the larger (usually positive) solution for $\phi$ is relevant. Hence, the square root term is added.
\begin{equation}
    \phi_{left}(d^*) = \textcolor{red}{-}\frac{d_f}{d_p}\tan{\alpha_t} + \sqrt{\left(\frac{d^*}{d_p}\right)^2 - \left(\frac{d_f}{d_p}\right)^2}
\end{equation}


\subsubsection{Special Case \texorpdfstring{$\phi_0 = \phi(d_f)$}{φ₀ = φ(dₓ)}}
The case of $d^* = d_f$ is interesting since this is the starting angle where one generally wants to start the undercut curve:
\begin{align}
\phi_0 = \phi(d_f) &=  \mypm\frac{d_f}{d_p}\tan{\alpha_t} \pm \sqrt{\cancel{\left(\frac{d_f}{d_p}\right)^2- \left(\frac{d_f}{d_p}\right)^2}}
\nonumber\\
&= \mypm\frac{d_f}{d_p}\tan{\alpha_t}
\end{align}

\subsection{Summary}
\begin{equation}
\begin{gathered}
\vb{r_{undercut}}(\phi) = \frac{1}{2} \left(
    \begin{bmatrix}
        a \\ b
    \end{bmatrix} \cos(\phi)
    + \begin{bmatrix}
        -b \\ a
    \end{bmatrix} \sin(\phi)
    + d_p\,\phi\; \begin{bmatrix}
        \sin(\phi) \\ - \cos(\phi)
    \end{bmatrix}
    \right)
\\
a = d_f \quad\quad b = \pm d_f\tan(\alpha_t)
\end{gathered}
\end{equation}

\begin{align}
    \phi(d^*) &= \mypm\frac{d_f}{d_p}\tan{\alpha_t} \pm \sqrt{\left(\frac{d^*}{d_p}\right)^2- \left(\frac{d_f}{d_p}\right)^2}
    \\
    \phi_{right}(d^*) &= \frac{d_f}{d_p}\tan{\alpha_t} - \sqrt{\left(\frac{d^*}{d_p}\right)^2 - \left(\frac{d_f}{d_p}\right)^2}
    \\
    \phi_{left}(d^*) &= \textcolor{red}{-}\frac{d_f}{d_p}\tan{\alpha_t} + \sqrt{\left(\frac{d^*}{d_p}\right)^2 - \left(\frac{d_f}{d_p}\right)^2}
    \\
    \phi_0 = \phi(d_f) &= \mypm\frac{d_f}{d_p}\tan{\alpha_t}
\end{align}

\paragraph{Note:} The $\mypm$ depends on which flank
\begin{itemize}
    \item $\textcolor{red}{+}$: Right flank $\Rightarrow$ $\phi$ is monotonically decreasing ($\phi_{start} > \phi_{end}$)\newline
        usually: $\phi_0 > 0 \qq{,} \phi_{end} < 0$ 
    \item $\textcolor{red}{-}$: Left flank $\Rightarrow$ $\phi$ is monotonically increasing ($\phi_{start} < \phi_{end}$)\newline
        usually: $\phi_0 < 0 \qq{,} \phi_{end} > 0$ 
\end{itemize}
