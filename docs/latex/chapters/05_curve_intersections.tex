\section{Curve Intersections}

\subsection{Newton-Raphson Method}
Finding analytical solutions for the intersections between the two curves is extremely difficult, so a numerical approach is used instead. Specifically, the Newton-Raphson method provides an iterative way to find roots of nonlinear equations.

\paragraph{1D Newton-Raphson}
For a single variable, the method iteratively refines an estimate of the root:
\begin{equation}
\begin{gathered}
f(x) \stackrel{!}{=} 0 \\
x^{k+1} = x^k - \frac{f(x^k)}{f'(x^k)}
\end{gathered}
\label{eq:1D_newton_raphson}
\end{equation}

\paragraph{2D Newton-Raphson}
For systems of two equations, the derivative is replaced by the Jacobian matrix. Given a vector function $\vb F$ and its Jacobian $\vb J$, the iteration becomes:
\begin{equation}
\begin{gathered}
    \vb F(\vb x) = \begin{bmatrix} f_1(x_1, x_2) \\ f_2(x_1, x_2) \end{bmatrix} \stackrel{!}{=} \vb 0
    \\[1ex]
    \vb J(\vb x) = \begin{bmatrix} J_{11} & J_{12} \\ J_{21} & J_{22} \end{bmatrix} = \begin{bmatrix} \pdv{f_1}{x_1} & \pdv{f_1}{x_2} \\ \pdv{f_2}{x_1} & \pdv{f_2}{x_2} \end{bmatrix}
    \\[1ex]
    \vb J^{-1} = \frac{1}{J_{11}J_{22} - J_{12}J_{21}} \begin{bmatrix}J_{22} & - J_{12} \\ -J_{21} & J_{11}\end{bmatrix}
    \\[1ex]
\vb x^{k+1} = \vb x^k - \vb J^{-1}(\vb x^k)\;\vb F(\vb x^k)
\end{gathered}
\label{eq:2D_newton_raphson}
\end{equation}

\subsection{Left–Right Involute Intersection}
In certain configurations, particularly when profile shift is applied, the left and right involute flanks of a tooth may intersect before reaching the dedendum circle. When this occurs, the involute curves must be truncated at their intersection point rather than extending to the dedendum. To determine this intersection, the positioned involute equations are set equal.

\subsubsection{System of Equations}
The right and left involutes are rotated by $-\gamma$ and $+\gamma$ respectively:
\begin{subequations}
\begin{align}
    &\vb{r_{inv,pos,right}}(\phi) = \vb{R}(-\gamma)\;\vb{r_{inv}}(\phi)
\nonumber\\[1ex]
    &\phantom{\vb{r_{inv,po}}}= \frac{d_b}{2} \begin{bmatrix}
        \phantom{-}\cos(\gamma)\;\cos(\phi) +\cos(\gamma)\;\phi \sin(\phi) + \sin(\gamma)\;\sin(\phi) - \sin(\gamma)\;\phi\cos(\phi)\\
        -\sin(\gamma)\;\cos(\phi) - \sin(\gamma)\;\phi \sin(\phi) + \cos(\gamma)\;\sin(\phi) - \cos(\gamma)\;\phi\cos(\phi)\\
    \end{bmatrix}
\\[2ex]
    &\vb{r_{inv,pos,left}}(\phi) = \vb{R}(\gamma)\;\vb{r_{inv}}(\phi)
\nonumber\\[1ex]
    &\phantom{\vb{r_{inv,po}}} = \frac{d_b}{2}
    \begin{bmatrix}
        \phantom{-}\cos(\phantom{-}\gamma)\;\cos(\phi) +\cos(\phantom{-}\gamma)\;\phi \sin(\phi) + \sin(-\gamma)\;\sin(\phi) - \sin(-\gamma)\;\phi\cos(\phi)\\
        -\sin(-\gamma)\;\cos(\phi) - \sin(-\gamma)\;\phi \sin(\phi) + \cos(\phantom{-}\gamma)\;\sin(\phi) - \cos(\phantom{-}\gamma)\;\phi\cos(\phi)\\
    \end{bmatrix}
\end{align}
\end{subequations}

\subsubsection{Equating Curves}
Setting the two curves equal yields two equations, one for each component.

The $x$-component gives:
\begin{equation}
\begin{aligned}
\cos(\gamma)\;\cos(\phi_r) &+ \cos(\gamma)\;\phi_r \sin(\phi_r) + \sin(\gamma)\;\sin(\phi_r) - \sin(\gamma)\;\phi_r\cos(\phi_r)\\
&= \cos(\gamma)\;\cos(\phi_l) +\cos(\gamma)\;\phi_l \sin(\phi_l) - \sin(\gamma)\;\sin(\phi_l) + \sin(\gamma)\;\phi_l\cos(\phi_l)
\end{aligned}
\end{equation}

The $y$-component gives:
\begin{equation}
\begin{aligned}
-\sin(\gamma)\;\cos(\phi_r) &- \sin(\gamma)\;\phi_r \sin(\phi_r) + \cos(\gamma)\;\sin(\phi_r) - \cos(\gamma)\;\phi_r\cos(\phi_r)\\
&= \sin(\gamma)\;\cos(\phi_l) + \sin(\gamma)\;\phi_l \sin(\phi_l) + \cos(\gamma)\;\sin(\phi_l) - \cos(\gamma)\;\phi_l\cos(\phi_l)
\end{aligned}
\end{equation}

\paragraph{Symmetry Argument}
Due to the symmetry of the two curves, the ansatz $\phi_r = -\phi_l = \phi$ is proposed. This can be verified by substituting into the $x$-component equation, which yields:
\begin{equation}
\begin{aligned}
    &\phantom{=}\,\;\cos(\gamma)\;\cos(\phi) + \cos(\gamma)\;\phi \sin(\phi) + \sin(\gamma)\;\sin(\phi) - \sin(\gamma)\;\phi\cos(\phi)
\\
    &= \cos(\gamma)\;\cos(\phi) + \cos(\gamma)\;\phi \sin(\phi) + \sin(\gamma)\;\sin(\phi) - \sin(\gamma)\;\phi\cos(\phi)
\end{aligned}
\end{equation}
Both sides are identical, confirming the symmetry assumption is valid.

\paragraph{Deriving the Intersection Condition}
Inserting $\phi_r = -\phi_l = \phi$ into the $y$-component equation:
\begin{equation}
\begin{aligned}
&\phantom{=}\,\; -\sin(\gamma)\;\cos(\phi) - \sin(\gamma)\;\phi \sin(\phi) + \cos(\gamma)\;\sin(\phi) - \cos(\gamma)\;\phi\cos(\phi)\\
&=
\phantom{-}\sin(\gamma)\;\cos(\phi) + \sin(\gamma)\;\phi \sin(\phi) - \cos(\gamma)\;\sin(\phi) + \cos(\gamma)\;\phi\cos(\phi)
\end{aligned}
\end{equation}

Taking the difference (RHS $-$ LHS $= 0$) and simplifying:

\begin{subequations}
\begin{align}
    2\sin(\gamma)\cos(\phi) + 2\sin(\gamma)\phi\sin(\phi) - 2\cos(\gamma)\sin(\phi) + 2\cos(\gamma)\phi\cos(\phi) &= 0\\
    \sin(\gamma)\cos(\phi) + \sin(\gamma)\phi\sin(\phi) - \cos(\gamma)\sin(\phi) + \cos(\gamma)\phi\cos(\phi) &= 0\\
    \sin(\gamma)\left[\cos(\phi) + \phi\sin(\phi)\right] - \cos(\gamma)\left[\sin(\phi) - \phi\cos(\phi)\right]&= 0
\end{align}
\vspace*{-2.5em}
\begin{align}
    \sin(\gamma)\left[\cos(\phi) + \phi\sin(\phi)\right] &= \cos(\gamma)\left[\sin(\phi) - \phi\cos(\phi)\right]\\
    \tan(\gamma) &= \frac{\sin(\phi) - \phi\cos(\phi)}{\cos(\phi) + \phi\sin(\phi)}\\
    \frac{\sin(\phi) - \phi\cos(\phi)}{\cos(\phi) + \phi\sin(\phi)} - \tan(\gamma) &=0
\end{align}
\end{subequations}

\subsubsection{Applying Newton-Raphson}
This means we need to solve the following equation with Newton-Raphson:
\begin{align}
f(\phi) = \frac{\sin(\phi) - \phi\cos(\phi)}{\cos(\phi) + \phi\sin(\phi)} - \tan(\gamma) &=0\\
\end{align}
For this we need to compute the derivative of $f(\phi)$ first.

\paragraph{Derivative of $f(\phi)$}
\begin{enumerate}[label=\alph*)]
    \item Split into components
        \begin{equation}
            \begin{gathered}
                f(\phi) = \frac{u(\phi)}{v(\phi)} -C\\
                \begin{aligned}
                    u(\phi) &=\sin(\phi) - \phi \cos(\phi) \\
                    v(\phi) &=\cos(\phi) + \phi \sin(\phi) \\
                    C &= \tan(\gamma)
                \end{aligned}
            \end{gathered}
        \end{equation}
    \item compute $u'(\phi)$
        \begin{equation}
            \begin{aligned}
                u'(\phi)    &= \cos(\phi) - \cos(\phi) + \phi\sin(\phi) \\
                            &= \phi\sin(\phi)
            \end{aligned}
        \end{equation}
    \item compute $v'(\phi)$
        \begin{equation}
            \begin{aligned}
                v'(\phi)    &= -\sin(\phi) + \sin(\phi) + \phi\cos(\phi) \\
                            &= \phi\cos(\phi)
            \end{aligned}
        \end{equation}
    \item apply quotient rule
        \begin{equation}
            \dv{}{\phi}\left[\frac{u(\phi)}{v(\phi)}\right] = \frac{u'\;v - u\;v'}{v^2}
        \end{equation}
        \begin{align}
            \dv{f(\phi)}{\phi} & = \frac{u'\;v - u\;v'}{v^2} \\
                &= \frac{
                    \phi\sin(\phi)\cdot\left[\cos(\phi) + \phi\sin(\phi)\right] - \phi\cos(\phi)\cdot\left[\sin(\phi) - \phi\cos(\phi)\right]
                    }
                    {\left[\cos(\phi) + \phi\sin(\phi)\right]^2}\\[1ex]
                &= \frac{
                    \cancel{\phi\sin(\phi)\cos(\phi)} + \phi^2\sin^2(\phi) \cancel{ -\phi\sin(\phi)\cos(\phi)} + \phi^2\cos^2(\phi)
                    }
                    {\left[\cos(\phi) + \phi\sin(\phi)\right]^2}\\[1ex]
                &= \frac{
                    \phi^2\cancelto{1}{\left[\sin^2(\phi) + \cos^2(\phi)\right]}
                    }
                    {\left[\cos(\phi) + \phi\sin(\phi)\right]^2}\\[1ex]
                &= \frac{\phi^2}{\left[\cos(\phi) + \phi\sin(\phi)\right]^2}
        \end{align}
\end{enumerate}

\paragraph{Newton-Raphson Iteration}
\begin{align}
    \phi_{n+1} &= \phi_n - \frac{F(\phi_n)}{F'(\phi_n)} \\
                &= \phi_n - \left(\frac{\sin(\phi_n) - \phi_n\cos(\phi_n)}{\cos(\phi_n) + \phi_n\sin(\phi_n)} - \tan(\gamma)\right)
                    \frac{\left[\cos(\phi_n) + \phi_n\sin(\phi_n)\right]^2}{\phi_n^2}\\[1ex]
                 &= \phi_n - \frac{\cos(\phi_n) + \phi_n\sin(\phi_n)}{\phi_n^2}\left(\sin(\phi_n) - \phi_n\cos(\phi_n) - \tan(\gamma)\;\left[\cos(\phi_n) + \phi_n\sin(\phi_n)\right]\right)
\end{align}
In a more compact form:
\begin{equation}
    \begin{gathered}
        \phi_{n+1} = \phi_n - \frac{a}{\phi_n^2}\left(b - \tan(\gamma)\;a\right)\\[1ex]
        a =\cos(\phi_n) + \phi_n\sin(\phi_n) \qq{and} b=\sin(\phi_n) - \phi_n\cos(\phi_n)
    \end{gathered}
\end{equation}


\subsection{Involute–Undercut Intersection}
In order to properly construct the tooth we need to find the intersection between involute and undercut which usually is just above the pitch circle. Using this we can draw the undercurve upto this intersection and start the involute from this intersection.
\subsubsection{System of Equations}
\begin{align}
    \vb{r_{inv}}(\phi_i) =
        \begin{bmatrix}
            x(\phi_i) \\
            y(\phi_i)
        \end{bmatrix}
    &= \frac{d_b}{2}
        \begin{bmatrix}
            \cos(\phi_i) \\
            \sin(\phi_i)
        \end{bmatrix}
        + \frac{d_b}{2}\phi_i
        \begin{bmatrix}
            \sin(\phi_i) \\
            -\cos(\phi_i)
        \end{bmatrix} \\
    \vb{r_{undercut}}(\phi_u) &= \frac{1}{2} \left(
        \begin{bmatrix} a \\ b \end{bmatrix} \cos(\phi_u)
        + \begin{bmatrix} -b \\ a \end{bmatrix}\sin(\phi_u)
        + d_p\,\phi_u\; \begin{bmatrix} \sin\phi_u \\ -\cos\phi_u \end{bmatrix}
    \right)\\[1ex]
    & a = d_f \quad\quad b = \mypm d_f\tan(\alpha_t) \nonumber
\end{align}
By looking at how the two curves are defined (see Figures~\ref{fig:involute_construction}~and~\ref{fig:undercut_generated_video}) one can see that the involute needs to be rotated by $\mypm(-\gamma)$ and the undercut curve needs to be rotated by $\mypm(-\gamma - \alpha_t)$. When finding the intersection this is equivalent to rotating the involute by $\mypm \alpha_t$.

Note: the red $\mypm$ refers to either the right flank ($\textcolor{red}{+}$) or left flank ($\textcolor{red}{-}$).

Thus the rotated involute equation is:
\begin{align}
    \vb{r_{inv,pos}}(\phi_i) &= \vb{R}(\mypm\alpha_t)\;\vb{r_{inv}}(\phi_i) \\[1ex]
    &=
    \frac{d_b}{2} \begin{bmatrix}
        \cos(\alpha_t) & -\sin(\mypm\alpha_t) \\ \sin(\mypm\alpha_t) & \cos(\alpha_t)
    \end{bmatrix}
    \begin{bmatrix}
        \cos(\phi_i) + \phi_i\sin(\phi_i)\\
        \sin(\phi_i) -\phi_i \cos(\phi_i)
    \end{bmatrix}
\\
    &=\frac{d_b}{2}\begin{bmatrix}
        \cos(\alpha_t)\cos(\phi_i) +\cos(\alpha_t)\phi_i\sin(\phi_i) -\sin(\mypm\alpha_t)\sin(\phi_i) +\sin(\mypm\alpha_t)\phi_i\cos(\phi_i) \\
        \sin(\mypm\alpha_t)\cos(\phi_i) +\sin(\mypm\alpha_t)\phi_i\sin(\phi_i) + \cos(\alpha_t)\sin(\phi_i) -\cos(\alpha_t)\phi_i\cos(\phi_i) \\
    \end{bmatrix}
\end{align}

\subsubsection{Equating Curves}
\begin{equation}
    \vb{r_{inv,pos}}(\phi_i) = \vb{r_{undercut}}(\phi_u)
\end{equation}
\begin{equation}
      \begin{aligned}
      & \frac{d_b}{2}\begin{bmatrix}
          c\cos(\phi_i) +c\phi_i\sin(\phi_i) -d\sin(\phi_i) +d\phi_i\cos(\phi_i) \\
          d\cos(\phi_i) +d\phi_i\sin(\phi_i) + c\sin(\phi_i) -c\phi_i\cos(\phi_i) \\
      \end{bmatrix}
      \\
      &\hspace{10em} =\frac{1}{2}\left(\begin{bmatrix}a \\ b\end{bmatrix}\cos\phi_u + \begin{bmatrix}
   -b \\ a\end{bmatrix}\sin\phi_u + d_p\phi_u\begin{bmatrix} \sin\phi_u \\ -
  \cos\phi_u\end{bmatrix}\right)
      \end{aligned}
\end{equation}
\begin{equation}
    d_b\begin{bmatrix}
        c\cos(\phi_i) +c\phi_i\sin(\phi_i) -d\sin(\phi_i) +d\phi_i\cos(\phi_i) \\
        d\cos(\phi_i) +d\phi_i\sin(\phi_i) + c\sin(\phi_i) -c\phi_i\cos(\phi_i) \\
    \end{bmatrix}
    =
    \begin{bmatrix}
        a \cos(\phi_u) - b\sin(\phi_u)+ d_p\phi_u\sin(\phi_u)\\
        b\cos(\phi_u) + a \sin(\phi_u) - d_p\phi_u\cos(\phi_u)
    \end{bmatrix}
\end{equation}

\paragraph{Equating in $\vb x$}
\begin{equation}
\begin{aligned}
    f_1 &= d_bc\cos(\phi_i) + d_bc\phi_i\sin(\phi_i) - d_bd\sin(\phi_i) + d_bd\phi_i\cos(\phi_i) \\
    &\phantom{=}- a\cos(\phi_u) + b\sin(\phi_u) - d_p\phi_u\sin(\phi_u)\\
    &= 0
\end{aligned}
\end{equation}

\paragraph{Equating in $\vb y$}
\begin{equation}
\begin{aligned}
    f_2 &= d_bd\cos(\phi_i) + d_bd\phi_i\sin(\phi_i) + d_bc\sin(\phi_i) -d_bc\phi_i\cos(\phi_i) \\
    &\phantom{=}-b\cos(\phi_u) - a \sin(\phi_u) + d_p\phi_u\cos(\phi_u) \\
    &= 0
\end{aligned}
\end{equation}

\subsubsection{Applying Newton-Raphson}

\paragraph{Jacobian Matrix}
Before we can numerically solve this using Newton-Raphson we first need to compute the Jacobian of this system of equations:
\begin{align}
    x_1 = \phi_i \qq{,} x_2=\phi_u \\[3ex]
    \vb J = \begin{bmatrix}\pdv{f_1}{x_1} & \pdv{f_1}{x_2} \\ \pdv{f_2}{x_1} & \pdv{f_2}{x_2}\end{bmatrix}
\end{align}

\begin{equation}
\begin{aligned}
    f_1 &= d_bc\cos(x_1) + d_bcx_1\sin(x_1) - d_bd\sin(x_1) + d_bdx_1\cos(x_1) \\
        &\phantom{=}- a\cos(x_2) + b\sin(x_2) - d_px_2\sin(x_2)
\end{aligned}
\end{equation}

\begin{equation}
\begin{aligned}
    f_2 &= d_bd\cos(x_1) + d_bdx_1\sin(x_1) + d_bc\sin(x_1) -d_bcx_1\cos(x_1) \\ 
        &\phantom{=}-b\cos(x_2) - a \sin(x_2) + d_px_2\cos(x_2)
\end{aligned}
\end{equation}

\begin{align}
    \pdv{f_1}{x_1} &= \cancel{-d_bc\sin(x_1) + d_bc\sin(x_1)} + d_bcx_1\cos(x_1) \cancel{- d_bd\cos(x_1) + d_bd\cos(x_1)} - d_bdx_1\sin(x_1)\\
        &= d_bcx_1\cos(x_1) - d_bdx_1\sin(x_1)\\
        &= d_bx_1[c\cos(x_1) - d\sin(x_1)]\\
            \pdv{f_1}{x_2} &= a\sin x_2 + b\cos x_2 - d_p \sin x_2 - d_px_2 \cos x_2 \\
            \pdv{f_2}{x_1} &= \cancel{-d_bd\sin(x_1) + d_bd\sin(x_1)} + d_bdx_1\cos(x_1) \cancel{+ db_c\cos(x_1) - d_bc\cos(x_1)} + d_bcx_1\sin(x_1)\\
        &= d_bdx_1\cos(x_1) + d_bcx_1\sin(x_1)\\
        &= d_bx_1[d\cos(x_1) + c\sin(x_1)]\\
            \pdv{f_2}{x_2} &= b\sin x_2 - a\cos x_2 + d_p\cos x_2 - d_px_2\sin x_2
\end{align}
Finally:
\begin{align}
    \vb J =
        \begin{bmatrix}
            d_bx_1[c\cos x_1 - d\sin x_1] & a\sin x_2 + b\cos x_2 - d_p \sin x_2 - d_px_2 \cos x_2\\
            d_bx_1[d\cos x_1 + c\sin x_1] & b\sin x_2+a\cos x_2 + d_p\cos x_2 - d_px_2\sin x_2 \\
        \end{bmatrix}
\end{align}

\paragraph{Inverse of Jacobian}
We compute the inverse of the Jacobian as:
\begin{align}
    \vb {J^{-1}} &= \frac{1}{J_{11}J_{22} - J_{12}J_{21}}
        \begin{bmatrix}
            J_{22} & - J_{12} \\
            -J_{21} & J_{11}
        \end{bmatrix}\\[2ex]
    J_{11} &= d_bx_1[c\cos x_1 - d\sin x_1] \\
    J_{12} &= a\sin x_2 + b\cos x_2 - d_p \sin x_2 - d_px_2 \cos x_2  \\
    J_{21} &= d_bx_1[d\cos x_1 + c\sin x_1] \\
    J_{22} &= b\sin x_2 - a\cos x_2 + d_p\cos x_2 - d_px_2\sin x_2
\end{align}

\paragraph{Newton-Raphson Iteration}
Finally the intersection can be found by iterating:
\begin{align}
    \begin{bmatrix}
        x_1 \\ x_2
    \end{bmatrix}^{(k+1)}
    = 
    \begin{bmatrix}
        x_1 \\ x_2
    \end{bmatrix}^{(k)}
    - \vb{J^{-1}} \cdot \begin{bmatrix} f_1 \\ f_2 \end{bmatrix}
\end{align}
where
\begin{align}
    x_1 = \phi_i \qq{,} x_2=\phi_u \\[3ex]
    f_1 &= d_bc\cos(x_1) + d_bcx_1\sin(x_1) - d_bd\sin(x_1) + d_bdx_1\cos(x_1)\nonumber\\
        &\phantom{=}- a\cos(x_2) + b\sin(x_2) - d_px_2\sin(x_2)\\[1ex]
    f_2 &= d_bd\cos(x_1) + d_bdx_1\sin(x_1) + d_bc\sin(x_1) -d_bcx_1\cos(x_1)\nonumber\\ 
        &\phantom{=}-b\cos(x_2) - a \sin(x_2) + d_px_2\cos(x_2) \\[3ex]
    \vb {J^{-1}} &= \frac{1}{J_{11}J_{22} - J_{12}J_{21}}
        \begin{bmatrix}
            J_{22} & - J_{12} \\
            -J_{21} & J_{11}
        \end{bmatrix}\\[2ex]
    J_{11} &= d_bx_1[c\cos x_1 - d\sin x_1] \\
    J_{12} &= a\sin x_2 + b\cos x_2 - d_p \sin x_2 - d_px_2 \cos x_2  \\
    J_{21} &= d_bx_1[d\cos x_1 + c\sin x_1] \\
    J_{22} &= b\sin x_2 - a\cos x_2 + d_p\cos x_2 - d_px_2\sin x_2
\end{align}

