\section{Tooth Geometry and Positioning of Curves}
\label{sec:tooth_geometry_positioning}
To construct a gear tooth, the two parametric curves (involute and undercut curve) need to be correctly positioned relative to each other and to the $x$-axis. The curves should be rotated such that mirroring them along the $x$-axis produces a valid tooth geometry with both a right and left flank.

\subsection{Tooth Thickness and Half Tooth Angle, \texorpdfstring{$\gamma$}{γ}}
\label{subsec:tooth_thickness}
The tooth thickness (in arc length) at the pitch circle, assuming \textbf{no profile shift}, is:
\begin{equation}
    s_0 = \frac{m\pi}{2}
\end{equation}

\begin{figure}[ht]
    \centering
    \includegraphics[width=\textwidth]{de-evolventenverzahnung-zahnrad-profilverschiebung-vergroesserung-zahndicke.jpg}
    \caption{Gemetric description of additional tooth width due to profile shift. Here $V=x\cdot m :=$ Absolute profile shift value. (\fullcite{profileshift})}
    \label{fig:profile_shift_geometry}
\end{figure}
The additional tooth width due to profile shift is more easily computed on the cutting rack as the linear addiional width corresponds to the additional arc length of the tooth on the gear. Figure~\ref{fig:profile_shift_geometry} geometrically shows how the additional tooth width, $\Delta s$ is computed:
\begin{equation}
    \Delta s = m\cdot x \cdot \tan(\alpha_n)
\end{equation}
Resulting in:
\begin{subequations}
\begin{align}
    s_0 &= \frac{p_0}{2} + \Delta s \\
        &= \frac{m\pi}{2} + m \cdot x \cdot \tan(\alpha_n) \\
        &= \frac{m}{2}\left(\pi + 2 \cdot x \cdot \tan(\alpha_n)\right)
\end{align}
\end{subequations}
Therefore, the angle spanned by half of the tooth at the pitch circle is:
\begin{subequations}
\begin{align}
    \gamma_0    &= \frac{s_0}{d_p} \\
                &= \frac{m\left(\pi + 2 \cdot x \cdot \tan(\alpha_n)\right)}{2\,d_p}
\end{align}
\end{subequations}

To get the angle spanned by half the tooth at the base circle, the angle spanned by the involute at $\phi(d_p)$ must be added.

The equation for obtaining the involute angle as derived in Section~\ref{subsec:involute_angle} is:
\begin{equation}
\theta(d^*) = \left[\left(\frac{d^*}{d_b}\right)^2 - 1\right]^{\tfrac{1}{2}} - \arctan\left(\left[\left(\frac{d^*}{d_b}\right)^2 - 1\right]^{\tfrac{1}{2}}\right)
\end{equation}

Inserting the relevant diameters:
\begin{subequations}
\begin{align}
\Delta \gamma &= \theta(d_p) - \cancelto{0}{\theta(d_b)} \\
    &= \theta(d_p) \\
    &= \left[\left(\frac{d_p}{d_b}\right)^2 - 1\right]^{\tfrac{1}{2}}
        - \arctan\left(\left[\left(\frac{d_p}{d_b}\right)^2 - 1\right]^{\tfrac{1}{2}}\right)
\end{align}
\end{subequations}

This results in a half tooth angle at the base circle of:

\begin{subequations}
\begin{align}
    \gamma &= \gamma_0+\Delta \gamma \\
           &= \frac{m\left(\pi + 2 \cdot x \cdot \tan(\alpha_n)\right)}{2\,d_p} + \left[\left(\frac{d_p}{d_b}\right)^2 - 1\right]^{\tfrac{1}{2}} - \arctan\left(\left[\left(\frac{d_p}{d_b}\right)^2 - 1\right]^{\tfrac{1}{2}}\right)
\end{align}
\end{subequations}

\subsection{Positioned Involute}
The involute simply needs to be rotated by half the tooth angle at the base circle:
\begin{subequations}
\begin{align}
    \vb{r_{inv,pos}}(\phi) &= \vb{R}(\mypm(-\gamma))\vb{r_{inv}}(\phi) \\[1ex]
    &= \begin{bmatrix}
            \cos(\gamma) & \sin(\mypm\gamma) \\
            -\sin(\mypm\gamma) & \cos(\gamma)
        \end{bmatrix}
        \left(
            \tfrac{d_b}{2} \begin{bmatrix}
                \cos(\phi) \\
                \sin(\phi)
            \end{bmatrix}
            + \tfrac{d_b}{2}\phi \begin{bmatrix}
                \sin(\phi) \\
                -\cos(\phi)
            \end{bmatrix}
        \right)
\\[1ex]
    &= \frac{d_b}{2} \begin{bmatrix}
            \cos(\gamma) & \sin(\mypm\gamma) \\
            -\sin(\mypm\gamma) & \cos(\gamma)
        \end{bmatrix}
        \begin{bmatrix}
            \cos(\phi) +\phi \sin(\phi)\\
            \sin(\phi) -\phi \cos(\phi)
        \end{bmatrix}
\\[1ex]
    &= \frac{d_b}{2} \begin{bmatrix}
            \phantom{-}\cos(\myppm\gamma)\;\cos(\phi) +\cos(\myppm\gamma)\;\phi \sin(\phi) + \sin(\mypm\gamma)\;\sin(\phi) - \sin(\mypm\gamma)\;\phi\cos(\phi)\\
            -\sin(\mypm\gamma)\;\cos(\phi) - \sin(\mypm\gamma)\;\phi \sin(\phi) + \cos(\myppm\gamma)\;\sin(\phi) - \cos(\myppm\gamma)\;\phi\cos(\phi)\\
        \end{bmatrix}
\end{align}
\end{subequations}

\subsection{Positioned Undercut Curve}

Due to the definition of the undercut curve (see Figure~\ref{fig:undercut_video_screenshot}), it needs to be rotated by both the half tooth angle at the base circle and the transversal pressure angle ($\alpha_t$).

\begin{subequations}
\begin{align}
    &\vb{r_{undercut,pos}}(\phi) = \vb{R}(\mypm(-\gamma - \alpha_t))\;\vb{r_{undercut}}(\phi)
\\[1ex]
    &= \begin{bmatrix}
            \cos(\gamma+\alpha_t) & \sin(\mypm(\gamma+\alpha_t)) \\
            -\sin(\mypm(\gamma+\alpha_t)) & \cos(\gamma+\alpha_t)
        \end{bmatrix}
        \;\frac{1}{2}\left(
            \vb A \cos(\phi) + \vb B \sin(\phi) + d_p\,\phi\;\vb t(\phi)
        \right)
\\[1ex]
    &= \frac{1}{2}\begin{bmatrix}
            \cos(\gamma+\alpha_t) & \sin(\mypm(\gamma+\alpha_t)) \\
            -\sin(\mypm(\gamma+\alpha_t)) & \cos(\gamma+\alpha_t)
        \end{bmatrix}
        \left(
            \begin{bmatrix}
                a \\ b
            \end{bmatrix} \cos(\phi)
            + \begin{bmatrix}
                -b \\ a
            \end{bmatrix} \sin(\phi)
            + d_p\,\phi\; \begin{bmatrix}
                \sin(\phi) \\ -\cos(\phi)
            \end{bmatrix}
        \right)
\\[1ex]
    &= \frac{1}{2} \begin{bmatrix}
            \cos(\gamma+\alpha_t) & \sin(\mypm(\gamma+\alpha_t)) \\
            -\sin(\mypm(\gamma+\alpha_t)) & \cos(\gamma+\alpha_t)
        \end{bmatrix}
        \begin{bmatrix}
            a\cos(\phi) - b \sin(\phi) + d_p\,\phi\sin(\phi) \\
            b\cos(\phi) + a \sin(\phi) - d_p\,\phi\cos(\phi)
        \end{bmatrix}
\\[1ex]
    &= \frac{1}{2}\bigg[\begin{array}{cc}
            \phantom{-}a\cos(\myppm\;\gamma+\alpha_t\;)\;\cos(\phi) 
            - b\cos(\myppm\;\gamma+\alpha_t\;)\;\sin(\phi)
                + d_p\cos(\myppm\;\gamma+\alpha_t\;)\;\phi\sin(\phi)\dots \\
            -a\sin(\mypm(\gamma+\alpha_t))\;\cos(\phi)
                +b\sin(\mypm(\gamma+\alpha_t))\;\sin(\phi)
                - d_p\sin(\mypm(\gamma+\alpha_t))\;\phi\sin(\phi)\dots
        \end{array}
\nonumber\\
        &\phantom{= \frac{1}{2}\bigg[}\begin{array}{cc}
            \hspace{1em}\dots + b\sin(\mypm(\gamma+\alpha_t))\;\cos(\phi) 
                + a\sin(\mypm(\gamma+\alpha_t))\;\sin(\phi)
                - d_p\sin(\mypm(\gamma+\alpha_t))\;\phi\cos(\phi)\\
            \hspace{1em}\dots+ b\cos(\myppm\;\gamma+\alpha_t\;)\;\cos(\phi) 
                + a\cos(\myppm\;\gamma+\alpha_t\;)\;\sin(\phi)
                - d_p\cos(\myppm\;\gamma+\alpha_t\;)\;\phi\cos(\phi)
        \end{array}\bigg]
\end{align}
\end{subequations}

